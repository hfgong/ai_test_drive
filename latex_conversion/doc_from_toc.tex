\documentclass{article}
\usepackage{geometry}
\geometry{a4paper, margin=1in}

\title{Table of Contents}
\author{}
\date{}

\begin{document}

\maketitle

\tableofcontents

\section*{Preface}

\section{Introduction and Preliminaries}
\subsection{The R Environment}
\subsection{Related Software and Documentation}
\subsection{R and Statistics}
\subsection{R and the Window System}
\subsection{Using R Interactively}
\subsection{An Introductory Session}
\subsection{Getting Help with Functions and Features}
\subsection{R Commands, Case Sensitivity, etc.}
\subsection{Recall and Correction of Previous Commands}
\subsection{Executing Commands From or Diverting Output to a File}
\subsection{Data Permanency and Removing Objects}

\section{Simple Manipulations; Numbers and Vectors}
\subsection{Vectors and Assignment}
\subsection{Vector Arithmetic}
\subsection{Generating Regular Sequences}
\subsection{Logical Vectors}
\subsection{Missing Values}
\subsection{Character Vectors}
\subsection{Index Vectors; Selecting and Modifying Subsets of a Data Set}
\subsection{Other Types of Objects}

\section{Objects, Their Modes and Attributes}
\subsection{Intrinsic Attributes: Mode and Length}
\subsection{Changing the Length of an Object}
\subsection{Getting and Setting Attributes}
\subsection{The Class of an Object}

\section{Ordered and Unordered Factors}
\subsection{A Specific Example}
\subsection{The Function tapply() and Ragged Arrays}
\subsection{Ordered Factors}

\section{Arrays and Matrices}
\subsection{Arrays}
\subsection{Array Indexing. Subsections of an Array}
\subsection{Index Matrices}
\subsection{The array() Function}
\subsubsection{Mixed Vector and Array Arithmetic. The Recycling Rule}
\subsection{The Outer Product of Two Arrays}
\subsection{Generalized Transpose of an Array}
\subsection{Matrix Facilities}
\subsubsection{Matrix Multiplication}
\subsubsection{Linear Equations and Inversion}
\subsubsection{Eigenvalues and Eigenvectors}
\subsubsection{Singular Value Decomposition and Determinants}
\subsubsection{Least Squares Fitting and the QR Decomposition}
\subsection{Forming Partitioned Matrices, cbind() and rbind()}
\subsection{The Concatenation Function, c(), with Arrays}
\subsection{Frequency Tables from Factors}

\end{document}
