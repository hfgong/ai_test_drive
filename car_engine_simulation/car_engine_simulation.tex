\documentclass{article}
\usepackage{amsmath}
\usepackage{amsfonts}
\usepackage{amssymb}
\usepackage{graphicx}
\usepackage{hyperref}

\title{Car Engine Dynamics and Fuel Consumption Simulation}
\author{Your Name}
\date{\today}

\begin{document}

\maketitle

\section{Introduction}
This document outlines the formulas used to simulate the dynamics and fuel consumption of a car engine. The model considers the engine angular velocity, fuel consumption rate, fuel efficiency, power output, torque, and time-varying factors such as gas valve open level and load torque.

\section{Formulas}

\subsection{Fuel Consumption Rate}
The fuel consumption rate, \(\dot{m}(t)\), depends on the time-varying gas valve open level, \(V(t)\), and the engine angular velocity, \(\omega(t)\):

\begin{equation}
\dot{m}(t) = k_{\text{fuel}} \cdot V(t) \cdot \omega(t)
\end{equation}

where \(k_{\text{fuel}}\) is a proportional constant for the fuel consumption rate.

\subsection{Fuel Efficiency}
The fuel efficiency, \(\eta_f(\omega(t))\), is a function of the engine angular velocity, \(\omega(t)\):

\begin{equation}
\eta_f(\omega(t)) = \eta_{\max} \cdot \left(1 - e^{-\alpha \cdot (\omega(t) - \omega_{\text{opt}})^2}\right)
\end{equation}

where \(\eta_{\max}\) is the maximum fuel efficiency, \(\alpha\) is a constant, and \(\omega_{\text{opt}}\) is the optimal angular velocity for maximum efficiency.

\subsection{Power Output}
The power output, \(P(t)\), is determined by the fuel consumption rate and the fuel efficiency:

\begin{equation}
P(t) = \dot{m}(t) \cdot \eta_f(\omega(t))
\end{equation}

\subsection{Total Torque}
The total torque produced by the engine, \(T_{\text{total}}(t)\), is related to the power output and angular velocity:

\begin{equation}
T_{\text{total}}(t) = \frac{P(t)}{(2 \pi \cdot \omega(t) / 60)}
\end{equation}

\subsection{Internal Friction Torque}
The internal friction torque, \(T_f(t)\), as a function of angular velocity is given by:

\begin{equation}
T_f(t) = k_{\text{friction}} \cdot \omega(t) + c_f
\end{equation}

where \(k_{\text{friction}}\) is the internal friction coefficient and \(c_f\) is a constant representing static friction.

\subsection{Net Torque}
The net torque, \(T_{\text{net}}(t)\), is the difference between the total torque, internal friction torque, and load torque:

\begin{equation}
T_{\text{net}}(t) = T_{\text{total}}(t) - T_f(t) - T_{\text{load}}(t)
\end{equation}

\subsection{Angular Acceleration}
The angular acceleration, \(\alpha(t)\), is determined by the net torque and the engine's moment of inertia, \(I\):

\begin{equation}
\alpha(t) = \frac{T_{\text{net}}(t)}{I}
\end{equation}

\subsection{Update Angular Velocity}
The angular velocity is updated using the angular acceleration over a small time step, \(dt\):

\begin{equation}
\omega(t+dt) = \omega(t) + \alpha(t) \cdot dt
\end{equation}

\section{Conclusion}
These formulas provide a comprehensive model for simulating car engine dynamics and fuel consumption, considering the time-varying nature of the gas valve open level and load torque. Adjust the constants and functions based on empirical data or specific engine characteristics for more accurate simulations.

\end{document}
